\documentclass[a4j,12pt]{jreport}
\usepackage[dvipdfmx]{hyperref}
\usepackage{pxjahyper}
\renewcommand{\bibname}{参考文献}
\usepackage{listings}
\usepackage{amsmath,amssymb,}

\makeatletter
    \renewcommand{\theequation}{
    \thesection.\arabic{equation}}
    \@addtoreset{equation}{section}
\makeatother

\lstset{
    basicstyle={\ttfamily},
    identifierstyle={\small},
    commentstyle={\smallitshape},
    keywordstyle={\small\bfseries},
    ndkeywordstyle={\small},
    stringstyle={\small\ttfamily},
    frame={tb},
    breaklines=true,
    columns=[l]{fullflexible},
    %numbers=left,
    xrightmargin=0zw,
    xleftmargin=0zw,
    numberstyle={\scriptsize},
    stepnumber=1,
    numbersep=1zw,
    lineskip=-0.5ex
}
\title{ {\LaTeX} 動作確認用ファイル}
\author{大矢 隼士}
\date{\today}
\begin{document}
    \maketitle
    \tableofcontents

    \chapter{VSCodeとWSLで\LaTeX }
        \section{WSLを使う利点}
            \begin{enumerate}
                \item Windowsの開発環境・アプリを利用できる
                
                SourceTreeとかATOKとかその他諸々

                \item Windowsの弱点であるCUI環境をカバーできる
                
                \item 管理が楽ちん
                
                こんな感じ\ref{apt-get}でできる\cite{sourceCode}\cite{sourceCode2}

            \end{enumerate}
            \begin{lstlisting}[caption=コマンド,label=apt-get]
$ sudo apt update
$ sudo apt upgrade
            \end{lstlisting}

        \section{WSL設定}
            \subsection{WSLのインストール}
                PowerShellを管理者権限で起動し, 下記を入力する. 
                \begin{lstlisting}[caption=PowerShell,label=ps]
PS> Enable-WindowsOptionalFeature -Online -FeatureName Microsoft-Windows-Subsystem-Linux
                \end{lstlisting}

            \subsection{Linuxのインストール}
                UbuntuとかWLinuxとかをMicrosoft Storeからインストールする.
            
            \subsection{\LaTeX のインストール}
                Linux(今回はDebian系)の初期設定を終わらせ、下記を入力する.
                \begin{lstlisting}[caption=Linuxのターミナル,label=bash]
$ sudo apt update
$ sudo apt upgrade
$ sudo apt install textive-full
                \end{lstlisting}
                

        \section{VSCodeの設定}

            
            \begin{enumerate}
                \item VSCodeをダウンロードする\cite{vscode}
                
                \item Shift + Ctrl + xで表示された検索窓を利用し \\
                LaTeX WorkShopをインストール
                
                \item \textasciitilde /AppData/Roaming/Code/User/settings.jsonを置き換える\cite{wsl}
            
            \end{enumerate}

            これでこのソースファイルがコンパイルできるはず.

    
    \chapter{数式テスト}
        を行うため自身の卒業論文\cite{ohya}を引用する.\\
        相対論的な粒子のエネルギーを$E$,運動量を$p$,光速を$c$,質量を$m$とすると
        \begin{equation}
            E^2 = c^2 p^2 + m^2 c^4
        \end{equation}
        が成り立つ.時空は$(1+n)$次元のユークリッド空間$\mathbb{R} \times \mathbb{R}^n$とし, 時間変数を$t$, 空間変数を$x$, とする. 対応原理により(1.1)は\\
        \begin{equation}
            E \to i\hbar \partial_t, \quad p\to -i\hbar\nabla 
        \end{equation}
        とすると, $\mathbb{R} \times \mathbb{R}^n$上の波動関数を$\Psi$ は,Klein-Gordon方程式
        \begin{equation}
            -\hbar^2 \partial_t^2\Psi = -c^2\hbar^2\Delta \Psi + m^2c^4\Psi 
        \end{equation}
        に従う.非線形Klein-Gordon方程式は,$V$を場のポテンシャルエネルギー密度とすると,$V'(|\Psi|^2)\Psi$の項を加えることにより
        \begin{equation}
            \frac{\hbar^2}{2mc^2} \partial_t^2\Psi - \frac{\hbar^2}{2m}\Delta\Psi + \frac{mc^2}{2}\Psi 
            + V'( |\Psi|^2 )\Psi = 0  \label{eq:1.1}
        \end{equation}
        の形をとる. 
        $mc^2t$と$\hbar$は同じ作用量(エネルギーと時間の積)の次元$[mc^2t]=[\hbar]=[action]$を持つので位相変調を施した波動関数
        \begin{equation}
            \psi(x,t)=\Psi(x,t)\exp(imc^2t/\hbar) \label{eq:1.2}
        \end{equation}
        を考える.因子$\exp(imc^2t/\hbar)$は位相変調を表す.(\ref{eq:1.1})に(\ref{eq:1.2})を代入すると

        \begin{equation}
            \begin{split}
                \frac{\hbar^2}{2mc^2} (\partial_t^2 \psi) \exp(- \frac{imc^2t}{\hbar}) - i\hbar (\partial_t \psi) \exp(- \frac{imc^2t}{\hbar}) - \frac{mc^2}{2} \psi \exp(- \frac{imc^2t}{\hbar}) \\ 
                - \frac{\hbar^2}{2m}\Delta \psi \exp(- \frac{imc^2t}{\hbar}) + \frac{mc^2}{2} \psi \exp(- \frac{imc^2t}{\hbar}) \\
                + V'(|\psi|^2) \psi \exp(- \frac{imc^2t}{\hbar}) =0 . \label{eq:1.a}
                \end{split}
        \end{equation}
        以上により(\ref{eq:1.a})から$\exp (- \frac{imc^2t}{\hbar})$を落とすと
        $\psi$は位相変調された非線形Klein-Gordon方程式
        \begin{equation}
            i\hbar \partial_t \psi + \frac{\hbar^2}{2m}\Delta \psi - V'(|\psi|^2)\psi 
            = \frac{\hbar^2}{2mc^2}\partial_t^2\psi \label{eq:1.3}
            \end{equation}
        を満たすことがわかる.

    \begin{thebibliography}{100}
        \bibitem{sourceCode} https://qiita.com/ta\_b0\_/items/2619d5927492edbb5b03
        \bibitem{sourceCode2} https://qiita.com/ayihis@github/items/c779e4ab5cd7580f1f87
        \bibitem{ohya} H. Ohya, Derivation of the incompressible Euler equation
        from the Klein-Gordon equation in the singular limit, 2010, graduation thesis.
        \bibitem{vscode} https://code.visualstudio.com/
        \bibitem{wsl} https://qiita.com/ryoi084/items/967315f3ad6a20734ffd
    \end{thebibliography}

\end{document}